\documentclass{beamer}
\usepackage{color}
\usepackage{bm}
\usepackage{graphicx}
\usepackage{hyperref}
\usepackage{listings}
\beamertemplatenavigationsymbolsempty
\definecolor{mygreen}{rgb}{0,0.6,0}
\definecolor{mymauve}{rgb}{0.58,0,0.82}
\lstset{ %
  backgroundcolor=\color{white},   % choose the background color; you must add \usepackage{color} or \usepackage{xcolor}
  basicstyle=\footnotesize,        % the size of the fonts that are used for the code
  commentstyle=\color{mygreen},    % comment style
  deletekeywords={...},            % if you want to delete keywords from the given language
  extendedchars=true,              % lets you use non-ASCII characters; for 8-bits encodings only, does not work with UTF-8
  frame=single,                    % adds a frame around the code
  keywordstyle=\color{blue},       % keyword style
  language=Python,                 % the language of the code
  rulecolor=\color{black},         % if not set, the frame-color may be changed on line-breaks within not-black text (e.g. comments (green here))
  showspaces=false,                % show spaces everywhere adding particular underscores; it overrides 'showstringspaces'
  showstringspaces=false,          % don't show spaces in strings
  stringstyle=\color{mymauve},     % string literal style
  title=\lstname                   % show the filename of files included with \lstinputlisting; also try caption instead of title
}
\hypersetup{
    colorlinks=true,
    urlcolor=blue
}

\graphicspath{ {./img/} {./charts/} }


\title{Django's System Check Framework}
\author{Adam Johnson - me@adamj.eu}
\date{13 July 2015}

\begin{document}


\maketitle


\begin{frame}[fragile]\frametitle{What is it?}

    \begin{center}
        \includegraphics[width=10cm]{checkit}
    \end{center}

    \begin{itemize}
        \item It checks your code!
    \end{itemize}

\end{frame}


\begin{frame}[fragile]\frametitle{What is it?}

    \begin{itemize}
        \item Added to Django in version 1.7
        \item Checks things after 'ready' time, such as models, deprecated features, and contrib apps
        \item Can be run alone as manage.py command
        \item Looks like this:
    \end{itemize}

        \begin{lstlisting}
$ ./manage.py check
System check identified some issues:

WARNINGS:

?: (1_6.W001) Some project unittests may not execute as
  expected.
    HINT: Django 1.6 introduced a new default test
    runner. It looks like this project was generated
    using Django 1.5 or earlier. You should ensure your
    tests are all running & behaving as expected. See https://docs.djangoproject.com/en/dev/releases/1.6/#discovery-of-tests-in-any-test-module
    for more information.
    \end{lstlisting}

\end{frame}


\begin{frame}[fragile]\frametitle{Why is it good?}

    \begin{center}
        \includegraphics[width=10cm]{celebrate}
    \end{center}

    \begin{itemize}
        \item Code quality!
    \end{itemize}

\end{frame}


\begin{frame}[fragile]\frametitle{Why is it good?}

    \begin{itemize}
        \item ``Beyond linting'' - code quality checks that can only be done at runtime, not by flake8
        \item Extensible - third party apps can provide checks too
        \item Runs on nearly every manage command - you can't forget to run it, unlike tests!
    \end{itemize}

\end{frame}


\begin{frame}[fragile]\frametitle{Using it}

    \begin{center}
        \includegraphics[width=10cm]{demo}
    \end{center}

    \begin{itemize}
        \item How do you use it?
    \end{itemize}

\end{frame}


\begin{frame}[fragile]\frametitle{Using it}

    \begin{itemize}
        \item Normally it does its work without you realizing, until you see an error and correct it
        \item However it's easy to write your own checks - the framework is very simple
        \item At YPlan, we already had some sanity checks in place as unit tests - converting them to checks has improved the situation
        \item I'll show you how now
    \end{itemize}

\end{frame}


\begin{frame}[fragile]\frametitle{An ex-unit test}

    \begin{itemize}
        \item An example old sanity check:
    \end{itemize}

    \begin{lstlisting}
class SupportedFeaturesTest(SimpleTestCase):
    """Check features that might not be supported in
       some OS versions"""
    def test_strftime_dash_strips_leading_zeroes(self):
        # Not every unix strftime has %P
        dt = datetime(2000, 10, 11, 2, 12)
        self.assertEqual(
            dt.strftime('%-I:%M %P'),
            '2:12 am'
        )
    \end{lstlisting}

    \begin{itemize}
        \item Bad! The test might not be run; or if it is and fails, its output is buried amongst all the other tests dependent on this failing
    \end{itemize}
\end{frame}



\begin{frame}[fragile]\frametitle{An ex-unit test}

    \begin{center}
        \includegraphics[width=10cm]{better}
    \end{center}

    \begin{itemize}
        \item Let's make it better!
    \end{itemize}

\end{frame}


\begin{frame}[fragile]\frametitle{Writing a check}

    \begin{itemize}
        \item A basic check:
    \end{itemize}

    \begin{lstlisting}
from django.core.checks import Error, Tags, register

@register(Tags.compatibility)
def check_system(app_configs, **kwargs):
    # app_configs is a list of AppConfig objects
    # kwargs is for future extension
    errors = []
    if bad_thing():
        errors.append(
            Error(
                "My error message",
                id='mycode.001'
            )
        )
    return errors
    \end{lstlisting}

    \begin{itemize}
        \item Put in \texttt{myapp/apps.py} next to your \texttt{MyAppConfig} class
    \end{itemize}

\end{frame}


\begin{frame}[fragile]\frametitle{Converting our unit test}

    \begin{itemize}
        \item Magicked into a mini check function:
    \end{itemize}

    \begin{lstlisting}
def check_system(app_configs, **kwargs):
    errors = []
    errors.extend(_check_strftime())
    return errors

def _check_strftime():
    errors = []
    dt = datetime(2000, 10, 11, 2, 12)
    if dt.strftime('%-I:%M %P') != '2:12 am':
        errors.append(Error(
            "strftime does not appear to support using "
            "%- to strip leading zeroes",
            id='yplan.E007'
        ))
    return errors
    \end{lstlisting}

\end{frame}


\begin{frame}[fragile]\frametitle{Thank you}

    \begin{center}
        \includegraphics[width=10cm]{clapping}
    \end{center}

    \begin{itemize}
        \item \url{me@adamj.eu} ; blog at adamj.eu/tech/
    \end{itemize}

\end{frame}


\end{document}
